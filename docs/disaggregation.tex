\documentclass[10pt]{article}         %% What type of document you're writing.

\usepackage{amsmath}
\usepackage{amsfonts}
\usepackage{amssymb}

\begin{document}

\section*{This describes how to disaggregate national FIES rates by urban and rural areas}

For a given country, we have the following variables:

\begin{itemize}
 \item $\omega$: The percentage of households in urban areas in a country with food insecurity (UNKNOWN)
 
 \item $\rho$: The percentage of households in rural areas in a country with food insecurity (UNKNOWN)
 
 \item $t$: The overall percentage of households in a country with food insecurity (KNOWN)
 
 \item $RUR$: The total population in rural areas (KNOWN)
 
 \item $URB$: The total population in urban areas (KNOWN)
 
 \item $TOT$: The total population (KNOWN)
 
 \item $ratio$: The ratio of the rates of urban food insecurity to rural food insecurity.  This is known at a regional level, and it is estimated that the same ratio holds at the individual country level.
\end{itemize}

Lets start from the following two assumptions/equations:

\begin{equation} 
	\omega / \rho = ratio 
	\label{eq1}
\end{equation}

\begin{equation} 
	\omega * URB + \rho * RUR = t * TOT
	\label{eq2}
\end{equation}
\newline
 
Now we can derive the value of $\omega$ from Equation \ref{eq1}.

\begin{gather*} 
\omega = \rho * ratio                   
\end{gather*}

Substituting that value into Equation \ref{eq2} yields:

\begin{gather*} 
\rho * ratio * URB + \rho * RUR = t * TOT                
\end{gather*}

Solving for $\rho$:

\begin{align*}
\rho * ratio * URB + \rho * RUR &= t * TOT     \\
\rho * (ratio * URB + RUR) &= t * TOT \\
\rho &= \frac{t * TOT}{ratio * URB + RUR}
\end{align*}

Then, given $\rho$, we can solve for $\omega$ with Equation \ref{eq2}.
 

\end{document}

